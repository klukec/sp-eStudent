% To je predloga za poročila o domačih nalogah pri predmetih, katerih
% nosilec je Blaž Zupan. Seveda lahko tudi dodaš kakšen nov, zanimiv
% in uporaben element, ki ga v tej predlogi (še) ni. Več o LaTeX-u izveš na
% spletu, na primer na http://tobi.oetiker.ch/lshort/lshort.pdf.
%
% To predlogo lahko spremeniš v PDF dokument s pomočjo programa
% pdflatex, ki je del standardne instalacije LaTeX programov.

\documentclass[a4paper,11pt]{article}
\usepackage{a4wide}
\usepackage{fullpage}
\usepackage[utf8x]{inputenc}
\usepackage[slovene]{babel}
\selectlanguage{slovene}
\usepackage[toc,page]{appendix}
\usepackage[pdftex]{graphicx} % za slike
\usepackage{setspace}
\usepackage{color}
\definecolor{light-gray}{gray}{0.95}
\usepackage{listings} % za vključevanje kode
\usepackage{hyperref}
\renewcommand{\baselinestretch}{1.2} % za boljšo berljivost večji razmak
\renewcommand{\appendixpagename}{Priloge}

\lstset{ % nastavitve za izpis kode, sem lahko tudi kaj dodaš/spremeniš
language=Python,
basicstyle=\footnotesize,
basicstyle=\ttfamily\footnotesize\setstretch{1},
backgroundcolor=\color{light-gray},
}

\title{Statična stran HTML/CSS in dinamičnost z JavaScript \\ Spletno programiranje}
\author{Matic Novak (63130164)}
\date{\today}

\begin{document}

\maketitle

\section{Moji podatki}

Matic Novak, 63130164, 2015/2016, spletno programiranje

\section{Ime in namen spletne strani}

Odločil sem se za izdelavo spletne aplikacije e-Študent, ker me tema privlači. Namen študijskega informacijskega sistema je informatizirati študijski proces. Odpravlja nekoč uveljavljene indekse, v katerih so študentje zbirali ocene. Vpis ocene je predstavljalo dodaten proces, saj se je kandidat moral dogovoriti s profesorjem in se fizično oglasiti pri njem z indeksom. Aplikacija zato omogoča vnos in celosten pregled nad študenti in njihovimi ocenami, nad profesorji, predmeti, ki se izvajajo. Tako študentom kot tudi profesorjem in študentskemu referatu olajša in pohitri poslovanje. V aplikaciji si bom močno prizadeval za dobro uborabniško izkušnjo, sicer bi ga bilo težje uvesti med uporabnike. Uporabljati ga bo moč na klasičnih računalnikih, kot tudi na telefonih in tablicah, kar je dan danes že nujna funkcionalnost spletne aplikacije.

\section{Ciljna publika in naprave}

Ciljna publika so študentje, profesorji in študentski referat. Slednji bodo aplikacijo večinoma uporabljali na klasičnih oz. prenosnih računalnikih, študentje pa najverjetneje na telefonih in tablicah. Aplikacija ni namenjena zabavi in sprostitvi, temveč resnemu delu in podpori poslovnemu procesu. Iz istega razloga bo vsebovala zgolj nekaj barv - prevladovanje rdeče in črne v več odtenkih. Povdared bo na uporabnosti.

\section{Zemljevid in struktura spletišča}

Uporabil bom hierarhično strukturo spletnišča. Z nekaj krovnih strani bo pot vodila še na podstrani, ki bodo pokrile celotno problemsko domeno. S tovrstno strukturo bom dosegel preglednost aplikacije in s tem dodal na uporabnosti.\newpage
Meni bo vertikalno pozicioniran ob levem robu strani. Spustni meni se bo prikazal ob kliku na krovno oznako, saj prehod z miškinim kazalcem na napravah, občutljivih na dotik ,ne pride v poštev. V primeru uporabe aplikacije na zelo majhnem zaslonu, bo meni na voljo šele ob kliku na ikono, da ne zaseda preveč prostora.

zemljevid in 2 žična okvirja !!!!!!!!!!

\section{HTML 5}

\begin{itemize}
\item V aplikaciji sem sevedal uprabil osnovne elemente, kot so odstavki (\texttt{<p>}), headingi (\texttt{<h1>}), odseki za novo vrstico (\texttt{<br />}), povezave z dodatnim oblikovanjem (\texttt{a class=""})...
\item Vstavil sem sezname, ki za alineje uporabljajo lastne slike (\texttt{<ul class="puscica" >}).
\item Za prikaz podatkov sem večkrat uporabil tabele, ki imajo posebej oblikovano glavo.
\item Brez elementov \texttt{<div>} in \texttt{<span>} ne gre.
\end{itemize}

Potrudil sem se z izbiro HTML 5 elementov:
\begin{itemize}
\item Obrazce sem ovil v značko \texttt{<form>}. Nekatera vnosna polja so opisana v znački \texttt{<label>}, druga pa opis vsebujejo kar v "placeholderju".
\item Obrazci vsebujejo zadnje elemente, kot so \texttt{input type="email", "password", "date", "range", "tel"}.
\item Vključil sem multimedijski element video.
\end{itemize}

\section{CSS}

\begin{itemize}
\item Slike sem združil v eno datoteko in jih razrezal v CSS-ju, saj s tem pridobim na hitrosti nalaganja strani (uporaba sprites).
\item Nesto
\end{itemize}

\end{document}
